\chapter*{内容梗概}
量子回路設計において,T-depthを削減することが非常に重要である.
T-depthを削減するようなMultiple Controlled Toffoli(MCT)ゲートの分解手法が研究されている.
既存のMCTゲートの分解では,分解を工夫することで,MCTゲートのT-depthを削減している.
複数のMCTゲートを分解する場合,ビットごとのT-depthに差が生じる.
既存のMCTゲートの分解手法では,ビットごとのT-depthを考慮していないため,
既存のMCTゲートの分解手法は最適でない可能性がある.
\par
本論文では,ビットごとのT-depthを考慮して,
MCTゲートの分解を行うことで,
T-depthを削減する手法を提案する.
提案手法では,T-depthの小さいビットを優先して使用し,
MCTゲートを分解することでT-depthを削減する.
また,後続のゲートを考慮して,
MCTゲートの分解を決定することで回路全体のT-depthを削減する.
提案手法を用いることで,
既存手法より,\rout{T-depthを平均17.1%削減できる}ことを確認した.
\begin{comment}
\begin{itemize}
\item 12/20 ver0 目次案作成
\item 12/23 ver1 第一章作成
\item 12/26 ver2 同期チェックの結果を反映:\rout{赤色で修正}
\item 12/27 一章,先生のコメントを反映:\gout{緑色で修正}
\item 12/28 chap2\_ver0:第二章作成
\item 12/30 chap2\_ver1:同期チェックを反映:\gout{緑色で修正}
\item 12/31 chap2\_ver2:先生のコメントを反映:\bout{青色で修正}
\item 1/5 chap3\_ver0:第三章作成
\item 1/7 chap3\_ver1:同期チェックを反映:\bout{青色で修正}
\item 1/13 chap3\_ver2:先生のコメントを反映:\mout{赤紫色で修正}
\item 1/14 chap3:先生のコメントを反映:\rout{赤色で修正}
\item 1/17 chap4\_ver0:第4章作成
\item 1/18 chap4\_ver1:同期チェックを反映:\rout{赤色で修正}
\item 1/21 chap4\_ver2:先生のコメントを反映:\bout{青色で修正・追加}
\item 1/22 chap5\_ver0:第5章作成
\item 1/22 chap5\_ver1:同期チェックを反映:\rout{赤色で修正}
\item 1/23 chap5:先生のコメントを反映:\bout{青色で修正}
\item 1/23 chap6\_ver0:第6章作成
\item 1/23 chap6\_ver1:同期チェックで特に指摘なし
\item 1/24 chap6:先生のコメントを反映:\rout{赤色で修正}
\end{itemize}
\end{comment}
