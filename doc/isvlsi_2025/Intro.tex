\section{Introduction}

%\begin{itemize}
%\item I will explain the field of research and its background.
% \begin{itemize}
% \item A quantum circuit that realizes quantum computing needs to be designed for each given logical function.
% \item Any logical function can be designed by combining multiple MCT gates.
% \item To execute MCT gates on an actual quantum computer, they need to be decomposed into a group of gates called Clifford+T.
% \item The number of stages of T gates that cannot be executed simultaneously in a quantum circuit is called T-depth.
% \item Since the execution time of T gates is longer than other Clifford+T gates, T-depth is used as an indicator of cost.
% \item There is a demand for a design method of quantum circuits that can reduce T-depth.
% \end{itemize}
% Quantum computers perform calculations using quantum superposition states in quantum mechanics\cite{nielsen2010quantum}.
% Quantum computers have quantum algorithms that can perform calculations faster than conventional computers.

% Typical quantum algorithms include Shor's algorithm\cite{shor1999polynomial}, which can quickly solve prime factorization, and Grover's algorithm\cite{grover1996fast}, which can quickly search unorganized databases.

% The discovery of these quantum algorithms has led to active research into quantum computers.

% It is also believed that physical limitations will make it difficult to improve the performance of integrated circuits\cite{2015Inte81:online}.

% For this reason, quantum computers that have calculation methods different from conventional calculation methods are attracting attention.

General quantum algorithms often include quantum circuits that
calculate Boolean functions, which must be
designed for each problem~\cite{yamashita2008ddmf}. Usually, we design
such sub-circuits using {\it Multiple Controlled Toffoli (MCT) gates}~\cite{barenco1995elementary} at the logic design level. These MCT gates are
then decomposed into physically realizable gates. For future
fault-tolerant quantum computations, we often consider the so-called
{\it Clifford+T} gate set as one of the promising physically realizable
gates.



Among the Clifford+T gates, the operation time of a T gate is much 
longer than that of other Clifford gates. For this reason, reducing
the {\it T-depth}, which is the number of T gates that cannot be
executed simultaneously, is an important issue when decomposing MCT
gates.
Therefore, many decomposition methods of MCT gates have been proposed to
 to reduce the T-depth~\cite{abdessaied2016technology,niemann2019t,baker2019decomposing}.
These existing decomposition methods require different number of 
auxiliary qubits and different T-depth; we can choose the best
decomposition method for {\it one} MPMC gate from the existing methods. 

In this paper, we reveal that the existing decomposition 
methods may not decompose many multiple MPMC gates with the smallest
T-depth because they do not consider the T-depth of each qubit when  
decomposing an MCT gate. Therefore, we propose a framework to choose 
appropriate auxiliary qubits and the possibly best decomposition
method with considering T-depth of each qubit before the MCT gate to
be decomposed.
Additionally, to account for the impact of decomposing one MCT
gate on the decomposition of subsequent gates, 
our framework 
utilizes {\it beam search} to select  the best possible decomposition for each MCT gate. 
Our experimental results confirm that the proposed method reduces the maximum T-depth
by an average of 17.1\% compared to the case of applying the existing
methods.


% The proposed method and the existing method \cite{abdessaied2016technology,niemann2019t,baker2019decomposing}
% were applied to the benchmark circuit \cite{wille2008revlib}, and a comparative experiment was performed.
% The maximum T-depth in the circuit after applying the method was used as the comparison indicator.
%It was confirmed that the proposed method reduces the maximum T-depth
%by an average of \rout{17.1}%\rout{compared to the case of applying
% the existing method. }


This paper is organized as follows.
Next section summarizes existing decomposition methods of MCT
gates. Then, Section~III proposes our method to decompose MCT gates 
with considering T-depth of each qubit. Section~IV shows our
experimental results to confirm the effectiveness of our
method. Finally, Section~V concludes the paper with our future work. 


% \item Explain the current state of research in this field.
%\begin{itemize}
% \item A method has been proposed to reduce the T-depth by decomposing MCT gates.
% \item In existing methods, the T-depth value for each bit is considered uniform.
%\end{itemize}
%\item Then explain what the problem to be solved (the problem addressed in this paper) is.
%\begin{itemize}
% \item In existing methods, the T-depth value for each bit is not considered.
% \item Using bits with a large T-depth first may worsen the T-depth.
% %\end{itemize}
% \par
% When multiple MCT gates are decomposed, differences arise in the T-depth value for each bit, but
% existing MCT gate decomposition
% methods\cite{abdessaied2016technology,niemann2019t,baker2019decomposing}
% do not consider this difference.  
% For this reason, in existing decomposition methods for MCT gates,
% when decomposing multiple MCT gates, the T-depth may increase by using bits with larger T-depth values first.
% %\item I will briefly introduce the key idea behind the solution.
% %\begin{itemize}
% % \item Calculate the T-depth for each bit.
% % \item Decompose the MCT gate by using bits with smaller T-depth values ​​first.
% % \item Decompose the MCT gate by using beam search to consider subsequent gates.
% %\end{itemize}
% Therefore, in this paper, we propose a method to reduce the T-depth by
% considering the T-depth of each qubit and using bits with smaller T-depth values first.

% When decomposing multiple MCT gates, greedily using bits with smaller T-depth values may increase the T-depth of the entire circuit.
% For this reason, it is necessary to consider subsequent gates when
% decomposing MCT gates.
% Therefore, in this paper, we also propose a method to reduce 
% the T-depth by decomposing MCT gates, taking into account subsequent
% gates using beam search.

