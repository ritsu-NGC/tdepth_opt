\section{Introduction}

%\begin{itemize}
%\item I will explain the field of research and its background.
% \begin{itemize}
% \item A quantum circuit that realizes quantum computing needs to be designed for each given logical function.
% \item Any logical function can be designed by combining multiple MCT gates.
% \item To execute MCT gates on an actual quantum computer, they need to be decomposed into a group of gates called Clifford+T.
% \item The number of stages of T gates that cannot be executed simultaneously in a quantum circuit is called T-depth.
% \item Since the execution time of T gates is longer than other Clifford+T gates, T-depth is used as an indicator of cost.
% \item There is a demand for a design method of quantum circuits that can reduce T-depth.
% \end{itemize}
Quantum computers perform calculations using quantum superposition states in quantum mechanics\cite{nielsen2010quantum}.
Quantum computers have quantum algorithms that can perform calculations faster than conventional computers.

Typical quantum algorithms include Shor's algorithm\cite{shor1999polynomial}, which can quickly solve prime factorization, and Grover's algorithm\cite{grover1996fast}, which can quickly search unorganized databases.

The discovery of these quantum algorithms has led to active research into quantum computers.

It is also believed that physical limitations will make it difficult to improve the performance of integrated circuits\cite{2015Inte81:online}.

For this reason, quantum computers that have calculation methods different from conventional calculation methods are attracting attention.

\par

General quantum algorithms include a part that calculates a logical function\cite{yamashita2008ddmf}.

A quantum circuit that calculates this logical function must be designed for each given logical function.

In quantum circuit design, logical functions are designed using Multiple Controlled Toffoli (MCT) gates\cite{barenco1995elementary}.

To realize the operation of MCT gates, it is necessary to decompose them into a group of gates called Clifford+T.

Among these Clifford+T gates is a gate called a T gate.

The operation time of a T gate is much longer than that of other Clifford+T gates.

For this reason, reducing the T-depth, which is the number of T gates that cannot be executed simultaneously, is an important issue in quantum circuit design.

Decomposition methods for MCT gates that reduce the T-depth are being researched.

%\item Explain the current state of research in this field.
%\begin{itemize}
% \item A method has been proposed to reduce the T-depth by decomposing MCT gates.
% \item In existing methods, the T-depth value for each bit is considered uniform.
%\end{itemize}
%\item Then explain what the problem to be solved (the problem addressed in this paper) is.
%\begin{itemize}
% \item In existing methods, the T-depth value for each bit is not considered.
% \item Using bits with a large T-depth first may worsen the T-depth.
%\end{itemize}
\par
When multiple MCT gates are decomposed, differences arise in the T-depth value for each bit, but
existing MCT gate decomposition methods\cite{abdessaied2016technology,niemann2019t,baker2019decomposing} do not consider this difference. 
For this reason, in existing decomposition methods for MCT gates,
when decomposing multiple MCT gates, the T-depth may increase by using bits with larger T-depth values first.
%\item I will briefly introduce the key idea behind the solution.
%\begin{itemize}
% \item Calculate the T-depth for each bit.
% \item Decompose the MCT gate by using bits with smaller T-depth values ​​first.
% \item Decompose the MCT gate by using beam search to consider subsequent gates.
%\end{itemize}
\gout{Therefore,} in this paper, we propose a method to reduce the T-depth by considering the T-depth value for each bit and
using bits with smaller T-depth values first.
\par
When decomposing multiple MCT gates, greedily using bits with smaller T-depth values may increase the T-depth of the entire circuit.
For this reason, it is necessary to consider subsequent gates when decomposing MCT gates.
\gout{Therefore,} in this paper, we also propose a method to reduce the T-depth by decomposing MCT gates, taking into account subsequent gates using beam search.
%\item Appeal what kind of (experimental) results were obtained (at the table of contents stage, this was a hopeful prediction)
% \begin{itemize}
% \item Experiments were performed in the case where the circuit was given uninitialized auxiliary bits and in the case where the auxiliary bits were given with 0.
% \item Comparison was performed using the maximum T-depth after decomposing all gates
% \item On average, XX\% of the T-depth was reduced from the existing method
% \end{itemize}
\par
The proposed method and the existing method \cite{abdessaied2016technology,niemann2019t,baker2019decomposing}
were applied to the benchmark circuit \cite{wille2008revlib}, and a comparative experiment was performed.
The maximum T-depth in the circuit after applying the method was used as the comparison indicator.
It was confirmed that the proposed method reduces the maximum T-depth by an average of \rout{17.1}%\rout{compared to the case of applying the existing method. }
% Average reduction rate of all experimental results.
%\item Chapter structure
% \begin{itemize}
% \item This paper consists of six chapters.
% \item Chapter 2 describes the basics of quantum circuits and quantum gates.
% \item Chapter 3 describes the existing decomposition method of MCT gates.
% \item Chapter 4 describes the method proposed in this paper.
% \item Chapter 5 describes the evaluation method of the proposed method, experimental results, and considerations.
% \item Chapter 6 summarizes this paper and describes future issues.
% \end{itemize}
\par
This paper consists of six chapters.
Chapter 2 describes the basics of quantum circuits, quantum gates, and beam search.
In Chapter 3, we present a decomposition method for MCT gates that reduces the existing T-depth.

In Chapter 4, we explain a decomposition method for MCT gates that takes into account the T-depth for each bit.

In Chapter 5, we present the evaluation method for the proposed method, experimental results, and discussion.

In Chapter 6, we summarize this research and discuss future issues.

%\end{itemize}
