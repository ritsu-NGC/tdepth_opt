\chapter{おわりに}
本論文では,
ビットごとのT-depthを考慮して,
MCTゲートを分解することで,
量子回路全体の最大のT-depthを削減する手法を提案した.
既存手法のMCTゲートの分解では,
ビットごとのT-depthを考慮していないため,
量子回路全体のT-depthが増加する場合が\rout{ある}.
提案手法では,ビットごとのT-depthを考慮して,
T-depthの小さなビットを優先して使用し,
MCTゲートを分解することで,
T-depthを削減\rout{できる}.
また,後続のゲートを考慮して,
ビームサーチを用いてMCTゲートを分解することで,
回路全体のT-depthを削減する手法を提案した.
ベンチマーク回路を用いて,
既存手法,提案手法それぞれを適用した回路の最大のT-depthを比較した.
\par
\rout{実験を行った}すべての回路で提案手法は既存手法よりも,最大のT-depthを削減でき,
\rout{最大のT-depthを平均17.1%}削減できることを確認した.
また,後続のゲートを考慮しビームサーチを用いて,
MCTゲートを分解することで,
計算時間は増加するがより多くのT-depthが削減できることを確認した.
\par
ビットごとのT-depthを考慮した,
手法2\rout{から}手法4の分解では,
MCTゲートをコントロールビット数の少ないMCTゲートに分解し,
再帰的にMCTゲートを分解する.
提案手法では,
このコントロールビット数の少ないMCTゲートを構成するビットを決定する際に,
T-depthの小さいビットを優先して配置している.
\rout{その際},
これらのMCTゲートを再帰的に分解するための補助ビットを考慮していない.
今後の課題としては,
手法2\rout{から}手法4において,より最大のT-depthを小さくするような,
各MCTゲートを構成するビットの決定方法を検討することが考えられる.
また,後続のゲートを考慮し,
ビームサーチを用いて分解を決定する手法では,
最大のT-depthが小さい分解が多く存在する場合,
局所解に陥り,
最大のT-depthが改善されない場合が考えられる.
そのため,
列挙する分解を確率的に選択することで,
より良い解を探索する手法の検討も今後の課題として考えられる.
 \begin{comment} 
 \begin{itemize}
   \item 得られた成果
   \item 結論
   \item 課題
     \begin{itemize}
       \item 再帰的に分解する場合にT-depth最小となる分解
       \item ビームサーチの改善
     \end{itemize}
\end{itemize}
\end{comment} 