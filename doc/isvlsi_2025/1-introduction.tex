\chapter{はじめに}

%\begin{itemize}
  %\item どのような分野の研究か,その背景について説明する.
  %  \begin{itemize}
 %     \item 量子計算を実現する量子回路は与えられた論理関数ごとに設計する必要がある.
  %    \item 複数のMCTゲートを組み合わせることで,任意の論理関数を設計できる.
   %   \item 実際の量子コンピュータでMCTゲートを実行するには,Clifford+Tと呼ばれるゲート群に分解する必要がある.
   %   \item 量子回路中の同時に実行できないTゲートの段数をT-depthと呼ぶ.
   %   \item Tゲートは実行時間が他のClifford+Tのゲートと比べ長いため,T-depthはコストの指標として用いられる.
   %   \item T-depthを削減するような量子回路の設計手法が求められている.
   % \end{itemize}
    量子計算機は,量子力学における量子の重ね合わせ状態を利用して計算する\cite{nielsen2010quantum}.
    量子計算機には従来の計算機と比較して高速に計算できる量子アルゴリズムが存在する.
    代表的な量子アルゴリズムとして,素因数分解を高速に解くことができるShorのアルゴリズム\cite{shor1999polynomial}
    や整理されていないデータベースを高速に探索できるGroverのアルゴリズム\cite{grover1996fast}などが挙げられる.
    これらの量子アルゴリズムの発見により,量子計算機の研究が盛んに行われるようになった.
    また,物理的な限界により,集積回路の性能向上は困難になると考えられている\cite{2015Inte81:online}.
    このため,従来の計算方式と異なる計算方式を持つ量子計算機が注目されている.
    \par
    一般的な量子アルゴリズムには論理関数を計算する部分が存在する\cite{yamashita2008ddmf}.
    この論理関数を計算する量子回路は与えられた論理関数ごとに設計する必要がある.
    量子回路設計では,Multiple Controlled Toffoli(MCT)ゲート\cite{barenco1995elementary}を用いて論理関数を設計する.
    MCTゲートの演算を実現するには,\rout{Clifford+Tと呼ばれるゲート群に分解する必要がある}\cite{zhou2000methodology}.
    このClifford+Tのゲート群の中に,Tゲートと呼ばれるゲートがある.
    Tゲートの操作時間は他のClifford+Tのゲートと\gout{比較して非常に}長い\cite{fowler2009high}.
    このため,量子回路中の\rout{同時に実行できないTゲートの段数であるT-depthを削減すること}が量子回路設計において重要な課題である\cite{amy2013meet,miller2014mapping,selinger2013quantum}.
    T-depthを削減するような,MCTゲートの分解手法が研究されている\cite{abdessaied2016technology,niemann2019t,baker2019decomposing}.

  %\item その分野の従来の研究状況について説明
  %\begin{itemize}
  %  \item MCTゲートの分解を工夫することでT-depthを削減する手法が提案されている.
  %  \item 既存手法では,ビットごとのT-depthの値を均一に考えている
  %\end{itemize}
  %\item そして,何が解決すべき問題(本論文で扱った問題)かを説明.
  %\begin{itemize}
  %  \item 既存手法では,ビットごとのT-depthの値を考慮していない
  %  \item T-dephが大きいビットを先に使用することでT-depthが悪化する場合がある
  %\end{itemize}
  \par
  \gout{複数のMCTゲートを分解するとビットごとのT-depthの値に差が生じるが,
  既存のMCTゲートの分解手法\cite{abdessaied2016technology,niemann2019t,baker2019decomposing}ではその差を考慮していない.}
  このため,既存のMCTゲートの分解手法では,
  複数のMCTゲートを分解する際に,T-depthの値が大きいビットを先に使用することでT-depthが\rout{増加}する場合がある.
  %\item どのようなアイデアで解決したか,キーアイデアを少しだけ披露
  %\begin{itemize}
  %    \item ビットごとのT-depthを計算
  %    \item T-depthが小さいビットを優先的に使用し,MCTゲートを分解
  %    \item ビームサーチを用いて,後のゲートを考慮してMCTゲートを分解
  %\end{itemize}
  \gout{そこで,}本論文では,ビットごとのT-depthの値を考慮し,
  T-depthの小さなビットを優先的に使用することでT-depthを削減する手法を提案する.
  \par
  複数のMCTゲートを分解する場合,貪欲にT-depthの小さなビットを使用すると,回路全体のT-depthが\rout{増加}する場合がある.
  そのため,MCTゲートを分解する際に後続のゲートを考慮する必要がある.
  \gout{そこで,}本論文では,ビームサーチを用いて後続のゲートを考慮し,MCTゲートを分解することでT-depthを削減する手法も提案する.
  %\item どのような(実験)結果が得られたか、アピール(目次案の段階では希望的予測)
   % \begin{itemize}
   %   \item 回路に初期化されていない補助ビットを与えた場合,0で初期化された補助ビットを与えた場合それぞれで実験.
   %   \item すべてのゲートを分解した後の最大のT-depthで比較
   %   \item 平均XX\%,既存手法からT-depthを削減
  %  \end{itemize}
  \par
  提案手法,既存手法\cite{abdessaied2016technology,niemann2019t,baker2019decomposing}を
  ベンチマーク回路\cite{wille2008revlib}に適用し,比較実験を行った.
  比較の指標には,\rout{手法適用後の回路における}最大のT-depthを用いた.
  提案手法は既存手法を適用した場合に比べ,最大のT-depthを平均\rout{17.1}%\rout{削減することが確認された.}
  % すべての実験結果の平均の削減率.
  %\item 章構成
  %  \begin{itemize}
  %    \item 本論文は6章で構成されている.
  %    \item 第2章で量子回路と量子ゲートの基礎知識について述べる.
  %    \item 第3章では,既存のMCTゲートの分解手法について述べる.
  %    \item 第4章では本論文の提案手法について述べる.
  %    \item 第5章では提案手法の評価方法と実験結果,考察について述べる.
  %    \item 第6章では本論文のまとめと,今後の課題について述べる.
  %  \end{itemize}
  \par
  本論文は6章から構成される.
  第2章では量子回路,量子ゲートとビームサーチの基礎的な知識について述べる.
  第3章では,既存のT-depthを削減するようなMCTゲートの分解手法について述べる.
  第4章では,ビットごとのT-depthを考慮したMCTゲートの分解手法について説明する.
  第5章では,提案手法の評価方法,実験結果と考察について述べる.
  第6章では,本研究のまとめと今後の課題について述べる.
%\end{itemize}